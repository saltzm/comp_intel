\documentclass[12pt,fleqn]{article}
\usepackage{amsmath}

\title{\bf Heuristic Solutions for the Forest Planning Problem}

\author{Matthew Saltz \\
        The University of Georgia \\
        Athens, Georgia 30602 U.S.A.}

\date{2013 April 21}

\begin{document}
\maketitle
\section{Introduction}
The goal of the forest planning (FP) problem is to maximize timber harvest
in a forest over a given number of time periods \cite{bettinger}. The forest consists
of a finite number of stands (non-overlapping regions), and a solution
to the problem consists of a harvest schedule over these stands.  In 
other words, given a target harvest for each time period, 
we want to assign a time slot to each of the stands such that our 
harvest in each time period is closest to the target.  Specifically,
we seek to find a schedule that minimizes the following equation:
\begin{equation}
\sum_{t = 1}^{nPeriods} (H_t - T)^2 
\label{eq:error}
\end{equation}
where $H_t$ is the harvest volume at time period $t$ and $T$ is the target
harvest per time period. However, due to the nature of the problem,
the schedule must abide by certain constraints. In order to prevent
erosion and other land maintenance issues, no two adjacent stands
may be scheduled for the same time period. This greatly increases
the complexity of the problem.  In this paper, we consider
two heuristic solutions, one being a genetic algorithm
and the other being a particle swarm optimization (PSO) algorithm. The
PSO solution is shown to outperform other attempts at using PSO for the
FP problem, though it is still inferior to other algorithms for this problem.

\section{Fitness}
Both algorithms attempt to minimize the same fitness function, which is
based on Equation~\ref{eq:error}. In addition, we add a penalty of 
$$ln(n+1) * nPeriods * T^2$$ where $n$ is the number of adjacency violations in 
the schedule (the number of adjacent stands that are scheduled for the same 
time period.  Because $nPeriods * T^2$ is the worst possible
fitness of a schedule with no violations, it is not possible for
a schedule with violations to have a higher fitness than a schedule without
violations. For the GA, this allows positive genetic material to be retained
from individuals with violations, while discouraging the propagation of 
those individuals themselves, and for the PSO, it guides particles away 
from adjacency violations and towards valid solutions.

\section{Experimentation}

\subsection{Genetic Algorithm Setup}

\subsection{PSO Setup}
The PSO setup for this problem was rather unusual.  With normal parameter settings,
the algorithm behaved poorly, obtaining solutions with fitness no better than
$10^8$.  However, with some adjustments, performance increased dramatically, so that
results were better than those reported in \cite{potter}.  The key changes
were to use a cognitive factor of 3 and a social factor of 1, and to lower the inertia value 
to 0.1.  Velocities were allowed to be between -2 and 2.  Results can be found in Figure~\ref{fig:results}.

\begin{figure}
\begin{centering}
\begin{tabular}{ | l | l | l | l |}
\hline
                & GA         & PSO     \\ \hline
Average Fitness & 1495       & 17,893,380     \\ \hline 
Best Fitness    & 1000       & 7012137    \\ \hline
Worst Fitness   & 50         & 30,671,841     \\ \hline
\end{tabular}
\caption{Comparison of fitness between GA and PSO techniques over 100 trials.}
\label{fig:results}
\end{centering}
\end{figure}

\begin{figure}
\begin{centering}
\begin{tabular}{ | l | l | l | l | l |}
\hline
                & Integer Programming &GA         & Standard PSO      \\ \hline
Time Period 1   & 33049.5             &1495       & 32523.75          \\ \hline 
Time Period 2   & 32933.6             &1000       & 33259.88          \\ \hline
Time Period 3   & 33399.4             &50         & 33133.29          \\ \hline
\end{tabular}
\caption{Harvests per time period for the best result of each search algorithm. The integer
solution was taken from \cite{bettinger}. The target harvest per time period was 34,467 MBF 
(million board feet) for each.}
\label{fig:harvests}
\end{centering}
\end{figure}

\begin{thebibliography}{9}
\bibitem{bettinger}
Bettinger, Pete, and Zhu, Jianping (2006) A New Heuristic Method for Solving Spatially Constrained Forest Planning Problems Based on Mitigation of Infeasibilities Radiating Outward from a Forced Choice. \emph{Silva Fennica}
\bibitem{potter}
Potter, W.D., Drucker, E., et al. (2008) Diagnosis, Configuration, Planning, and Pathfinding: Experiments in Nature-Inspired Optimization. \emph{Natural Intelligence for Scheduling, Planning and Packing Problems} pp. 267-294 
\end{thebibliography}


\end{document}
