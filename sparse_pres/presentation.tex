\documentclass{beamer}
\usepackage{graphicx}
\usepackage{amsmath}
\usetheme{Warsaw}
\usecolortheme{lily}



\begin{document}
\title{ "Optimal Bundling Strategies for Cable Companies"} 
\author{Derek Ponticelli}
\date{\today} 

\frame{\titlepage} 

\frame{\frametitle{Table of contents}\tableofcontents} 


\section{Problem Overview} 
\frame{\frametitle{How can cable tv and internet providers maximize profits?} 
 Unique product and cost structures require cable providers to investigate complex pricing policies


\begin{columns}[c]
\column{2in}

\begin{itemize}
\item Multiproduct monopolist with substitutable products
\item Capacity constraint creates a nearly vertical cost curve
\item Bundling, Usage Based Pricing, 3-Part Tariffs
\end{itemize}

\column{2in}
\begin{figure}[h]
%\includegraphics[width=1.5in]{Vertical_Costs.jpg}
\caption{Aggregate Demand and Marginal Cost Curves for Internet Providers}
\label{Vertical Costs}
\end{figure}
\end{columns}


}


\section{Literature Review} 
\frame{\frametitle{Literature Review}
\begin{itemize}
\item Adams and Yellen (1976)
	\begin{itemize}
	\item Provided examples illustrating cases when pure or mixed bundling could outperform separate pricing models
	\end{itemize}
\item McAfee, McMillan, and Whinston (1989)
	\begin{itemize}
	\item Develop reservation price model for a multiproduct monopolist
	\item Consumers purchase at most 1 unit of either good and value them independently
	\item MMW prove that mixed bundling is always weakly better than pure bundling and investigate necessary and sufficient conditions for mixed bundling to outperform separate pricing as well
	\end{itemize}
\item Venkatesh and Kamakura (2003)
	\begin{itemize}
	\item Extend MMW to consider special cases of multiproduct monopolists selling complements and substitutes
	\end{itemize}
\end{itemize} 
}


\section{Model Specification} 
\subsection{Basic Model}
\frame{\frametitle{Basic Model Specification}
\begin{columns}[c]

\column{3in}
\begin{itemize}
\item Multiproduct monopolist sells two substitutes and bundle at prices $p_1, p_2, p_b$
\item Consumers' valuations $(v_1,v_2)$ for the goods fall on continuous uniform distributions from 0 to 1
\item Consumers purchase at most one unit and have marginal utility from good i:
	$
	mu_{i} (v_i)= 
	\begin{cases}
	v_i & : q_j=0 \\
	\beta*v_i & :q_j=1
	\end{cases}
	$
\item Monopolist seeks to maximize profits while constrainting internet consumption below some cap, $c$. 
\end{itemize} 

\column{2in}
\begin{figure}[h]
%\includegraphics[width=1in]{unshaded_general_picture.jpg}
\caption{5 Consumer Types}
\label{Unshaded Consumer Regions}

%\includegraphics[width=1in]{shaded_general_picture.jpg}
\caption{Internet Users}
\label{Shaded Internet Consumers}
\end{figure}

\end{columns}
}

\subsection{Special Cases}
\frame{\frametitle{$\beta = 0$ Page 1}

\begin{columns}[c]

\column{2.5in}
\begin{itemize}
\item $\beta = 0$ implies perfect crowdout so that no one will purchase both goods
\item Without a capacity constraint, max profit $= \frac{2}{3 \sqrt{3}} \approx .385$ and is achieved when $p_1=p_2=\frac{1}{\sqrt{3}} \approx .577$
\item This means a cap will only bind if it is $\leq \frac{1}{3}$
\item Green area $= \frac{1}{2} - p_1 + p_2 +\frac{p^2_1}{2}- p_1 p_2$
\item Red area $=\frac{1}{2} + p_1 -p_2 -\frac{p^2_1}{2}$
\end{itemize}

\column{2.5in}
\begin{figure}[h]
%\includegraphics[width=1.5in]{beta_zero_general.jpg}
\caption{$p_1=.5, p_2=.5$}
\label{beta0_general}
\end{figure}
\begin{itemize}
\item Revenue $= \frac{p_1}{2}+\frac{p_2}{2}-p^2_1 - p^2_2 +2 p_{1} p_{2}-\frac{3}{2} p^2_1 p_2 + \frac{p^3_1}{2}$
\end{itemize}
\end{columns}
}

\frame{\frametitle{$\beta = 0$ Page 2}
\begin{itemize}
\item We want to maximize $\frac{p_1}{2}+\frac{p_2}{2}-p^2_1 - p^2_2 +2 p_{1} p_{2}-\frac{3}{2} p^2_1 p_2 + \frac{p^3_1}{2}+\lambda (c - ( \frac{1}{2} - p_1 + p_2 +\frac{p^2_1}{2}- p_1 p_2))$ where $c \in (0,\frac{1}{3})$
\item So, the optimal $p_1$ and $p_2$ are implicitly defined by the solution to the system: 
	\begin{itemize}	
	\item $(1-p_1) \left[\frac{1}{2}-2 p_1+2 p_2-3 p_1 p_2+\frac{3}{2} p^2_1 \right] = (p_1 - p_2 -1)\left[\frac{1}{2} - 2  p_2 + 2 p_1 - \frac{3}{2} p^2_1\right]$
	\item $p_2 = \frac{c-\frac{1}{2} + p_1 - \frac{p^2_1}{2}}{1-p_1}$
	\end{itemize}
\end{itemize}
\begin{figure}[h]
\begin{minipage}[b]{.3\linewidth}
\centering
%\includegraphics[width=1.5in]{beta_zero_p1.jpg}
\caption{$P_1$}
\label{Beta Zero $P_1$}
\end{minipage}
\quad
\begin{minipage}[b]{.3\linewidth}
\centering
%\includegraphics[width=1.5in]{beta_zero_p2.jpg}
\caption{$P_2$}
\label{Beta Zero $P_2$}
\end{minipage}
\quad
\begin{minipage}[b]{.3\linewidth}
\centering
%\includegraphics[width=1.5in]{beta_zero_profit.jpg}
\caption{Profit}
\label{Beta Zero Profit}
\end{minipage}
\end{figure}
}



\frame{\frametitle{$\beta = 1$ Page 1}
\begin{columns}[c]

\column{3in}
\begin{itemize}
\item $\beta = 1$ implies that there is no crowd out effect and marginal utilities are constant
\item In absence of a bundle, optimal pricing strategy is $p_1 = p_2 = \frac{1}{2}$, profit $=\frac{1}{2}$, and cap would bind at $c=\frac{1}{2}$
\item With a bundle, optimal pricing strategy is $p_1 = p_2 =\frac{1}{27} (2\sqrt{61}-1) \approx .54, p_b = \frac{11}{54}+\frac{5}{54}\sqrt{61} \approx .926$
\item Then, the maximum revenue is $\frac{182}{2187}+\frac{122}{2187}\sqrt{61} \approx .51891$ and the cap would bind at $c \approx .530083$
\end{itemize}

\column{2in}
\begin{figure}[h]
%\includegraphics[width=.9in]{beta_one_no_bundle.jpg}\
\caption{No Bundle}
\label{beta1_no_bundle}

%\includegraphics[width=.9in]{beta_one_bundle.jpg}
\caption{Bundle}
\label{beta1_bundle}

\end{figure}
\end{columns}
}

\frame{\frametitle{$\beta = 1$ Page 2}
\begin{columns}[c]

\column{3.5in}
\begin{itemize}
\item TV consumer area (red)= $(1-p_2)(p_b-p_2)$
\item Bundle consumer area (blue)= $(p_1 + p_2 - p_b)(2-p_1-p_2)+(1-p_1)(1-p_2)$
\item Internet consumer area (green) = $(1-p_1)(p_b-p_1)$
\item $R(p_1,p_2,p_b) = 2 p_b (p_1+p_2-p^2_1-p^2_1)+p_b-p^2_1-p^2_2+p^3_1+p^3_2+p^2_b (p_1+p_2) - p_1 p_2 p_b$
\item Our goal is to maximize: $$\pi(p_1,p_2,p_b,\lambda) = R(p_1,p_2,p_b) + \lambda\left[c-1-p_2+p_b+p_1 p_2- p_2 p_b +p^2_2\right]$$ where $c \in (0,.530084)$
\end{itemize}

\column{1.5in}
\begin{figure}[h]
%\includegraphics[width=1in]{beta_one_bundle.jpg}
\caption{Bundle}
\label{beta1_bundle}
\end{figure}

\end{columns}

}

\frame{\frametitle{$\beta =1$ Page 3}
 We can use the first order conditions to establish a system of 4 equations and 4 unknowns to find the following as function of $c$:

\begin{columns}[c]
\column{2in}
\begin{figure}[h]
%\includegraphics[width=1.5in]{beta1_profit.jpg}
\caption{Profit}
\label{beta1_profit}
%\includegraphics[width=1.5in]{beta1_p2.jpg}
\caption{$P_2$}
\label{beta1_p2}
\end{figure}

\column{2in}
\begin{figure}[h]
%\includegraphics[width=1.5in]{beta1_p1.jpg}
\caption{$P_1$}
\label{beta1_p1}
%\includegraphics[width=1.5in]{beta1_pb.jpg}
\caption{$P_b$}
\label{beta1_pb}
\end{figure}

\end{columns}
}

\subsection{General Model Results}
\frame{\frametitle{General Model Page 1}

\begin{itemize}
\item The region of consumers who purchase the bundle can take on many different forms depending on the value of $\beta$ and the relative bundle price
\item "Skinny" Cone: $\beta > \frac{1}{2} , \beta (p_1+p_2) \leq p_b < (1-\beta) p_1 + \beta p_2+ 2 \beta -1$
\item "Fat" Cone: $\beta>\frac{1}{2}, p_b < \beta (p_1+p_2)$
\item $\beta = 0$ Case: $\beta> \frac{1}{2}, p_b > (1-\beta) p_1 + \beta p_2+ 2 \beta -1$
\end{itemize}

\begin{figure}[h]
\begin{minipage}{.28\linewidth}
\centering
%\includegraphics[width=.9in]{general_skinny_cone.jpg}
\caption{"Skinny" }
\label{skinny}
\end{minipage}
\begin{minipage}{.28\linewidth}
\centering
%\includegraphics[width=.9in]{general_fat_cone.jpg}
\caption{"Fat" }
\label{fat}
\end{minipage}
\begin{minipage}{.28\linewidth}
\centering
%\includegraphics[width=.9in]{beta_zero_general.jpg}
\caption{$\beta = 0 $ }
\label{beta0_general}
\end{minipage}
\end{figure}

}


\frame{\frametitle{General Model Page 2}

\begin{itemize}
\item Reverse "Skinny" Cone: $\beta < \frac{1}{2} , \beta (p_1+p_2) > p_b \geq (1-\beta) p_1 + \beta p_2+ 2 \beta -1$
\item Reverse "Fat" Cone: $\beta<\frac{1}{2}, p_b <  (1-\beta) p_1 + \beta p_2+ 2 \beta -1$
\item $\beta = 0$ Case: $\beta< \frac{1}{2}, p_b >  \beta (p_1+p_2) $
\end{itemize}

\begin{figure}[h]
\begin{minipage}{.28\linewidth}
\centering
%\includegraphics[width=.9in]{general_reverse_skinny.jpg}
\caption{"Skinny"}
\label{Rskinny}
\end{minipage}
\begin{minipage}{.28\linewidth}
\centering
%\includegraphics[width=.9in]{general_reverse_fat.jpg}
\caption{"Fat"}
\label{Rfat}
\end{minipage}
\begin{minipage}{.28\linewidth}
\centering
%\includegraphics[width=.9in]{beta_zero_general.jpg}
\caption{$\beta = 0 $ }
\label{beta0_general}
\end{minipage}
\end{figure}

}


\frame{\frametitle{General Model Page 3}

\begin{columns}[c]

\column{3in}
\begin{itemize}
\item Parallel Regions: $\beta = \frac{1}{2} , \beta (p_1+p_2) > p_b $
\item $\beta = 0$ Case: $\beta = \frac{1}{2}, p_b \geq  \beta (p_1+p_2) $
\vspace{.2in}
\item As $\beta $ changes, the monopolist may prefer to offer the bundle to different consumers
\item When solving for general solutions in terms of $\beta$ and $c$, one must be careful in dealing with the number of choices for the monopolist
\end{itemize}

\column{2in}
\begin{figure}[h]
%\includegraphics[width=.9in]{general_parallel_lines.jpg}
\caption{Parallel}
\label{parallel}
%\includegraphics[width=.9in]{beta_zero_general.jpg}
\caption{$\beta = 0 $ }
\label{beta0_general}
\end{figure}

\end{columns}

}

\frame{\frametitle{General Model Page 4}
 Numerical results from maximizing profits when $\beta = .8$ leave many questions unanswered

\begin{columns}[c]
\column{2in}
\begin{figure}[h]
%\includegraphics[width=1.5in]{skinny_cone_top_profit.jpg}
\caption{Profit}
\label{beta1_profit}
%\includegraphics[width=1.5in]{skinny_cone_top_p2.jpg}
\caption{$P_2$}
\label{beta1_p2}
\end{figure}

\column{2in}
\begin{figure}[h]
%\includegraphics[width=1.5in]{skinny_cone_top_p1.jpg}
\caption{$P_1$}
\label{beta1_p1}
%\includegraphics[width=1.5in]{skinny_cone_top_pb.jpg}
\caption{$P_b$}
\label{beta1_pb}
\end{figure}

\end{columns}
}

\section{Possible Research Extensions}
\frame{\frametitle{Next Steps and Future Goals}
\begin{itemize}
\item Investigate how demand theoretically shifts as monopolists increase internet prices
\item Reach analytic conclusions for a general $\beta > \frac{1}{2}$ and $\beta < \frac{1}{2}$
	\begin{itemize}
	\item Or possibly, numeric conclusions after further exhausting possibilities
	\end{itemize}
\item Anticipating a decline in $\beta$, predict dynamic pricing strategy for monopolists
\item Introducing a more complex consumption funtion to allow for non-linear pricing strategy (3 part tariffs)
\item Empirical testing, matching model predictions with data, looking for demand shifts
\end{itemize}
}

\end{document}

